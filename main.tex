\UseRawInputEncoding
%
% Module 4 Chapter 8 Homework
% CSC160-C00: Computer Science I (C++)
% Author: Ashton Hellwig
% Date: 19 April 2020
%


\documentclass[a4paper, 11pt]{article}
  % Packages
  \usepackage[latin1,utf8]{inputenc}  % Encoding
  \usepackage[english]{babel}         % Internationalization
  \usepackage{times}                  % Times New Roman font
  \usepackage{soul}                   % Highlighting
  \usepackage{hyperref}               % Links (internal and external)
  \usepackage{fancyhdr}               % Headers and footers
  \usepackage[dvipsnames]{xcolor}     % Text Colors
  \usepackage{listings}               % Code Snippets
  \usepackage[section]{algorithm}     % For TOC support
  \usepackage{algpseudocode}          % Algorithmic notation environments
  \usepackage{enumitem}               % Ordered lists
  \usepackage{geometry}               % Page layout
  \usepackage{graphicx}               % Image support
  \usepackage[toc, page]{appendix}    % Appendix
  \usepackage{setspace}               % Paragraph and line spacing
  \usepackage{bookmark}               % Required for appendix
  \usepackage{adjustbox}              % Required for appendix
  \usepackage{csquotes}               % Required for appendix
  \usepackage{amsthm}                 % Theorem environments
  \usepackage{array}                  % Arrays
  \usepackage{makecell}               % Table helpers
  \usepackage{amsmath}                % Mathematical symbols
  \usepackage[fleqn]{mathtools}       % Mathematical environments
  \usepackage{amssymb}                % Misc. symbols for logic and math
  \usepackage{relsize}                % Relative Sizing
  \usepackage{multicol}               % Multi-figure displays (grid)
  \usepackage{etoolbox,refcount}      % Required for MDFramed
  \usepackage{parcolumns}             % Paragraph grids
  \usepackage{mdframed}               % Colored box environments
  \usepackage{float}                  % Floating Environments 
  \usepackage{aliascnt}
  \usepackage[                        % Bibliography management
    backend=biber,%
    style=apa%
  ]{biblatex}
  \usepackage{mathtools}
  \usepackage{empheq}
  \usepackage[skins,theorems]{tcolorbox}

  % Bibliography Setup
  \addbibresource{main.bib}
  \newcommand{\CiteSection}[2]{%
    (\autocite{#1}, ~\S {#1})%
  }

%   \UseRawInputEncoding

  % Tables
  \renewcommand\theadalign{bc}
  \renewcommand\theadfont{\bfseries}
  \renewcommand\theadgape{\Gape[4pt]}
  \renewcommand\cellgape{\Gape[4pt]}

  % Lists
  \newcounter{countitems}
  \newcounter{nextitemizecount}
  \newcommand{\setupcountitems}{%
    \stepcounter{nextitemizecount}%
    \setcounter{countitems}{0}%
    \preto\item{\stepcounter{countitems}}%
  }
  \makeatletter
  \newcommand{\computecountitems}{%
    \edef\@currentlabel{\number\c@countitems}%
    \label{countitems@\number\numexpr\value{nextitemizecount}-1\relax}%
  }
  \newcommand{\nextitemizecount}{%
    \getrefnumber{countitems@\number\c@nextitemizecount}%
  }
  \newcommand{\previtemizecount}{%
    \getrefnumber{countitems@\number\numexpr\value{nextitemizecount}-1\relax}%
  }
  \makeatother
  \newenvironment{AutoMultiColItemize}{%
  \ifnumcomp{\nextitemizecount}{>}{3}{\begin{multicols}{2}}{}%
  \setupcountitems\begin{itemize}}%
  {\end{itemize}%
  \unskip\computecountitems\ifnumcomp{\previtemizecount}{>}{3}{\end{multicols}}{}}



  % Colors
  \newcommand{\commentstylecolor}{\color{Gray}}
  \newcommand{\keywordstylecolor}{\color{MidnightBlue}}
  \newcommand{\stringstylecolor}{\color{ForestGreen}}
  \newcommand{\questioninput}{\color{Red}}
  \newcommand{\answertcolor}{\color{Green}}
  \newcommand{\myanswer}{\answertcolor{\hl}}

  % Symbols
  \newcommand{\answerflow}{\rotatebox[origin=c]{180}{$\Lsh$}}
  \newcommand{\toanswer}{\mathlarger{\mathlarger{\answerflow}}\quad}

  % Math
  \newcommand{\highlight}[1]{%
    \colorbox{green!50}{$\displaystyle#1$}}

  % Image Directory
  \graphicspath{ {screenshots/} }


  % Hyperlink Setup
  \hypersetup{
    colorlinks = true,
    urlcolor = blue,
    linkcolor = blue
  }


  % Algorithm Settings
  \newcommand{\pluseq}{\mathrel{{+}{=}}}
  \newcommand{\minuseq}{\mathrel{{-}{=}}}


  % Syntax-Highlighting for Code Snippets
  \lstset{
    backgroundcolor=\color{white},
    breaklines=true,%
    captionpos=b,%
    frame=tlrb,%
    columns=fixed,%
    tabsize=2,%
    numbers=left,%
    showstringspaces=false,%
    commentstyle=\commentstylecolor,%
    keywordstyle=\keywordstylecolor,%
    stringstyle=\stringstylecolor%
  }
  \lstset{literate=
  {á}{{\'a}}1 {é}{{\'e}}1 {í}{{\'i}}1 {ó}{{\'o}}1 {ú}{{\'u}}1
  {Á}{{\'A}}1 {É}{{\'E}}1 {Í}{{\'I}}1 {Ó}{{\'O}}1 {Ú}{{\'U}}1
  {à}{{\`a}}1 {è}{{\`e}}1 {ì}{{\`i}}1 {ò}{{\`o}}1 {ù}{{\`u}}1
  {À}{{\`A}}1 {È}{{\'E}}1 {Ì}{{\`I}}1 {Ò}{{\`O}}1 {Ù}{{\`U}}1
  {ä}{{\"a}}1 {ë}{{\"e}}1 {ï}{{\"i}}1 {ö}{{\"o}}1 {ü}{{\"u}}1
  {Ä}{{\"A}}1 {Ë}{{\"E}}1 {Ï}{{\"I}}1 {Ö}{{\"O}}1 {Ü}{{\"U}}1
  {â}{{\^a}}1 {ê}{{\^e}}1 {î}{{\^i}}1 {ô}{{\^o}}1 {û}{{\^u}}1
  {Â}{{\^A}}1 {Ê}{{\^E}}1 {Î}{{\^I}}1 {Ô}{{\^O}}1 {Û}{{\^U}}1
  {œ}{{\oe}}1 {Œ}{{\OE}}1 {æ}{{\ae}}1 {Æ}{{\AE}}1 {ß}{{\ss}}1
  {ű}{{\H{u}}}1 {Ű}{{\H{U}}}1 {ő}{{\H{o}}}1 {Ő}{{\H{O}}}1
  {ç}{{\c c}}1 {Ç}{{\c C}}1 {ø}{{\o}}1 {å}{{\r a}}1 {Å}{{\r A}}1
  {€}{{\euro}}1 {£}{{\pounds}}1 {«}{{\guillemotleft}}1
  {»}{{\guillemotright}}1 {ñ}{{\~n}}1 {Ñ}{{\~N}}1 {¿}{{?`}}1
}
  \newenvironment{alltt}{\ttfamily}{\par}
  \lstMakeShortInline[language=c++]|

  % Page Configuration
  %% Style
  \pagestyle{fancy}

  %% Layout
  \geometry{%
    a4paper,%
    top=2.5cm,%
    bottom=2.5cm,%
    left=2.5cm,%
    right=2.5cm%
  }
  \setlength{\headheight}{15pt}
  \setlength{\floatsep}{9pt}
  \setlength{\parindent}{2em}
  \setlength{\parskip}{0.3em}
  \renewcommand{\baselinestretch}{0.75}

  %% Title page
  \title{Module 4 Chapter 8 Homework}
  \author{Ashton Hellwig}
  \date\today
  \setcounter{tocdepth}{3}

  % MDFramed Config
  \mdfdefinestyle{AnswerFrame}{%
    linecolor=black,
    outerlinewidth=2pt,
    roundcorner=20pt,
    innertopmargin=4pt,%
    innerbottommargin=4pt,%
    innerrightmargin=4pt,%
    innerleftmargin=4pt,%
    leftmargin = 4pt,%
    rightmargin = 4pt,%
    backgroundcolor=green!20%
  }


 \input{tex/floatingequations}


% Document Content
\begin{document}
  % Title Page
  \maketitle
  \tableofcontents
  \lstlistoflistings
  \newpage

  % Question 1
  \section{Question 1}
    Determine whether the following array declarations are valid. If a
      declaration is invalid, explain why.

    \begin{enumerate}[label=\Alph*.]
      \item |string customers[];|
      \item |int numArray[50];|
      \item |const int SIZE = 30; double list[20 - SIZE];|
      \item |int length = 50; double list[length - 50];|
      \item |int ids[-30];|
      \item |colors[30] string;|
    \end{enumerate}

    % Question 1 Solution
    \subsection{Solution}
      \begin{enumerate}[label=\alph*.]
        \item \begin{mdframed}[style=AnswerFrame]
          \textbf{Invalid} -- Static array declarations require a
            size specifier or to be initialized.
          \end{mdframed}
        \item \begin{mdframed}[style=AnswerFrame]
          \textbf{Valid}
          \end{mdframed}
        \item \begin{mdframed}[style=AnswerFrame]
          \textbf{Invalid} -- Cannot declare an array with a negative size.
          \end{mdframed}
        \item \begin{mdframed}[style=AnswerFrame]
          \textbf{Valid}
          \end{mdframed}
        \item \begin{mdframed}[style=AnswerFrame]
          \textbf{Invalid} -- Cannot declare an array with a negative size.
          \end{mdframed}
        \item \begin{mdframed}[style=AnswerFrame]
          \textbf{Invalid} -- |typename| (|std::string|) should come before the declaration /
            |string| is a reserved word.
          \end{mdframed}
      \end{enumerate}


  % Question 2
  \section{Question 2}
    Determine whether the following array declarations are valid. If a declaration 
      is valid, determine the size of the array.

    \begin{enumerate}[label=\Alph*.]
      \item |int list[] = {18, 13, 14, 16};|
      \item |int x[10] = {1, 7, 5, 3, 2, 8};|
      \item |double y[4] = { 2.0, 5.0, 8.0, 11.0, 14.0};|
      \item |double lengths[] = { 8.2, 3.9, 6.4, 5.7, 7.3};|
      \item |int list[7] = {12, 13, , 14, 16, , 8};|
      \item |string names[8] = {"John", "Lisa", "Chris", "Katie"};|
    \end{enumerate}

    % Question 2 Solution
    \subsection{Solution}
      \begin{enumerate}[label=\alph*.]
        \item \begin{mdframed}[style=AnswerFrame]
          $\text{placeholder}$
          \end{mdframed}
        \item \begin{mdframed}[style=AnswerFrame]
          $\text{placeholder}$
          \end{mdframed}
        \item \begin{mdframed}[style=AnswerFrame]
          $\text{placeholder}$
          \end{mdframed}
        \item \begin{mdframed}[style=AnswerFrame]
          $\text{placeholder}$
          \end{mdframed}
        \item \begin{mdframed}[style=AnswerFrame]
          $\text{placeholder}$
          \end{mdframed}
        \item \begin{mdframed}[style=AnswerFrame]
          $\text{placeholder}$
          \end{mdframed}
      \end{enumerate}

  \section{Question 3}
    What is the output of the following C++ code?

    \begin{lstlisting}[language=c++,caption={Question 3 Program}]
#include <iostream>
using namespace std;
int main() {
  int list1[5];
  int list2[15];

  for (int i = 0; i < 5; i++)
    list1[i] = i * i - 2;
  cout << "list1: ";

  for (int i = 0; i < 5; i++)
    cout << list1[i] << " ";
  cout << endl;

  for (int i = 0; i < 5; i++) {
    list2[i] = list1[i] * i;
    list2[i + 5] = list1[4 - i] + i;
    list2[i + 10] = list2[9 - i] + list2[i];
  }
  cout << "list2: ";

  for (int i = 0; i < 7; i++)
    cout << list2[i] << " ";
  cout << endl;

  return 0;
}
    \end{lstlisting}

    % Question 3 Solution
    \subsection{Solution}
      \begin{lstlisting}[language=bash,caption={Question 3 Solution}]
Placeholder.
      \end{lstlisting}

    

  \newpage
  \section{Question 4}
    What is the output of the following program?
    
    \begin{lstlisting}[language=c++,caption={Question 4 Program}]
#include <iostream>
using namespace std;
int main() {
  int j;
  int one[5];
  int two[10];
  for (j = 0; j < 5; j++)
    one[j] = 5 * j + 3;

  cout << "One contains: ";
  for (j = 0; j < 5; j++)
    cout << one[j] << " ";
  cout << endl;
  for (j = 0; j < 5; j++) {
    two[j] = 2 * one[j] - 1;
    two[j + 5] = one[4 - j] + two[j];
  }

  cout << "Two contains: ";
  for (j = 0; j < 10; j++)
    cout << two[j] << " ";
  cout << endl;

  return 0;
}
    \end{lstlisting}
    
    % Question 4 Solution
    \subsection{Solution}
      \begin{lstlisting}[language=bash,caption={Question 4 Solution}]
Placeholder.
      \end{lstlisting}


  \newpage
  \section{Question 5}
    Assume that you have the following statements:

    \begin{lstlisting}[language=c++]
char name[21];
char yourName[21];
char studentName[31];
    \end{lstlisting}
    
    Mark the following statements as valid or invalid. If invalid, explain why.
    
    \begin{enumerate}[label=\Alph*.]
      \item Placeholder.
      \item Placeholder.
      \item Placeholder.
      \item Placeholder.
      \item Placeholder.
      \item Placeholder.
      \item Placeholder.
      \item Placeholder.
    \end{enumerate}
    
    % Question 5 Solution
    \subsection{Solution}
      \begin{enumerate}[label=\alph*.]
        \item \begin{mdframed}[style=AnswerFrame]
          $\text{placeholder}$
          \end{mdframed}
        \item \begin{mdframed}[style=AnswerFrame]
          $\text{placeholder}$
          \end{mdframed}
        \item \begin{mdframed}[style=AnswerFrame]
          $\text{placeholder}$
          \end{mdframed}
        \item \begin{mdframed}[style=AnswerFrame]
          $\text{placeholder}$
          \end{mdframed}
        \item \begin{mdframed}[style=AnswerFrame]
          $\text{placeholder}$
          \end{mdframed}
        \item \begin{mdframed}[style=AnswerFrame]
          $\text{placeholder}$
          \end{mdframed}
        \item \begin{mdframed}[style=AnswerFrame]
          $\text{placeholder}$
          \end{mdframed}
        \item \begin{mdframed}[style=AnswerFrame]
          $\text{placeholder}$
          \end{mdframed}
      \end{enumerate}


  % Bibliography
  \newpage
  \nocite{textbook}
  \printbibliography[%
    heading=bibintoc,%
    title={Works Consulted}%
  ]{}
  
%   Appendix
%   \newpage
%   \appendix
%   \section{Question 1}
%     Placeholder.
\end{document}
